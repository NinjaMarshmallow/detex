%% Author: Daniel Kaplan
%% Subject: Correlation

Using your general knowledge about the world, think about the
relationship between these variables:
\begin{itemize}
\item \VN{speed} of a bicyclist.
\item \VN{steepness} of the road, a quantitative variable 
measured by the grade (rise over
run). 0 means flat, + means uphill, $-$ means downhill.
\item \VN{fitness} of the rider, a categorical variable with three
levels: unfit, average, athletic.
\end{itemize}

On a piece of paper, sketch out a graph of speed versus steepness for reasonable models
of each of these forms:
\begin{enumerate}
\item Model 1: \model{\VN{speed}}{1 + \VN{steepness}}

\begin{AnswerText}
Imagine that positive steepness means uphill, and negative steepness
is downhill. 
As the hill gets steeper uphill, bicycle speed gets slower.  So this
model would be a line that slopes negatively.
\end{AnswerText}

\item Model 2: \model{\VN{speed}}{1 + \VN{fitness}}

\begin{AnswerText}
Increased fitness leads to higher speed, so the line of speed against fittness will slope upwards.
\end{AnswerText}

\item Model 3: \model{\VN{speed}}{1 + \VN{steepness}+\VN{fitness}}

\begin{AnswerText}
Speed will go down with greater steepness (uphill) and speed will go
up with greater fitness.
\end{AnswerText}

\item Model 4: \model{\VN{speed}}{1 + \VN{steepness}+\VN{fitness} +
    \VN{steepness}:\VN{fitness}}

\begin{AnswerText}
Compared to Model 3, what's new here is the interaction term between
steepness and fitness.  Presumably, more fit people don't slow down as
much when they encounter a hill, so the interaction should reduce the
effect of steepness.
\end{AnswerText}
\end{enumerate}