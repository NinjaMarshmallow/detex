%% Author: Daniel Kaplan
%% Subject: adjustment and covariates,

{\em Time Magazine} reported the results of a poll of people's
opinions about the U.S. economy in July 2008.  The results are summarized
in the graph.

\bigskip
\centerline{\includegraphics[width=3in]{../Figures/time-poll-july-2008.png}}
[Source: {\em Time}, July 28, 2008, p. 41]
\bigskip

The variables depicted in the graph are:
\begin{itemize}
\item \VN{Pessimism}, as indicated by agreeing with the statement that
the U.S. was a better place to live in the 1990s and will continue to
decline.
\item \VN{Ethnicity}, with three levels: White, African American, Hispanic.
\item \VN{Income}, with five levels.
\item \VN{Age}, with four levels.
\end{itemize}

Judging from the information in the graph, which of these statements
best describes the model \model{\VN{pessimism}}{\VN{income}}?
\begin{MultipleChoice}
\correct{\VN{Pessimism} declines as incomes get higher.}
\wrong{\VN{Pessimism} increases as incomes get higher.}
\wrong{\VN{Pessimism} is unrelated to income.}
\end{MultipleChoice}

Again, judging from the information in the graph, which of these
statements best describes the model \model{\VN{pessimism}}{\VN{age}}?
\begin{MultipleChoice}
\correct{\VN{Pessimism} is highest in the 18-29 age group.}
\wrong{\VN{Pessimism} is highest in the 64 and older group.}
\wrong{\VN{Pessimism} is lowest among whites.}
\wrong{\VN{Pessimism} is unrelated to age.}
\end{MultipleChoice}

Poll results such as this are almost always reported using just one
explanatory variable at a time, as in this graphic.  However, it can
be more informative to know the effect of one variable while {\em
adjusting for} other variables.  For example, in looking at the
connection between \VN{pessimism} and \VN{age}, it would be useful to
be able to untangle the influence of \VN{income}.

\begin{comment}

Imagine that the poll were reported in terms of a model 
\model{\VN{pessimism}}{\VN{age} + \VN{income}}. Keep in mind
that there is a strong correlation between \VN{age} and \VN{income}, with
people in the 18-29 age group typically having smaller incomes than
people in older age groups.

Which of the following is true about the \VN{age} coefficients from
such a model?
\begin{MultipleChoice}
\wrong{They would be exactly the same as in the
\model{\VN{pessimism}}{\VN{age}}?} 
\correct{They would display the relationship between \VN{pessimism}
and \VN{age} adjusting for \VN{income}.}
\wrong{They would display the relationship between \VN{pessimism}
and \VN{income} adjusting for \VN{age}.}
\end{MultipleChoice}

How do you think the conclusions suggested by the poll might be
different if the results were presented in terms of the model 
\model{\VN{pessimism}}{\VN{age} + \VN{income}} rather than the two
separate models \model{\VN{pessimism}}{\VN{age}} and 
\model{\VN{pessimism}}{\VN{income}}?  \TextEntry

\begin{AnswerText}
Taking into account that younger people tend to have lower incomes,
and that lower income people tend to be more pessimistic, it is likely
that the model \model{\VN{pessimism}}{\VN{age} + \VN{income}} would
show a different relationship between \VN{pessimism} and \VN{age}
than the model \model{\VN{pessimism}}{\VN{age}}.  For example,
Simpson's paradox might apply and it could be that, when adjusting for
income, younger people are {\em less} pessimistic than older people.

Unfortunately, it's not possible to reconstruct the multi-variable
models from the single-variable summaries that are typical of polls.
It would be helpful if results were presented after adjusting for the
other variables involved.
\end{AnswerText}

\end{comment}
